%& -job-name=slides
\documentclass{beamer}
\beamertemplatenavigationsymbolsempty

\title
    [Wordpress]
    {Wordpress}

\author
    [Ondrej Sika]
    {Ondrej Sika\\\texttt{ondrej@ondrejsika.com}}

\institute
    [Czechitas]
    {Czechitas\\\url{www.czechitas.cz}}

\date
    {01. 3. 2015}


\begin{document}

\maketitle

\begin{frame}
    {Co je Wordpress}

    Nejpopularnejsi redakcni system postaveny na PHP.
\end{frame}


\begin{frame}
    {Na co pouzit Wordpress}

    \begin{itemize}
        \item Osobni web
        \item Firemni prezentace
        \item Blogy
    \end{itemize}
\end{frame}


\begin{frame}
    {Na co NEpouzit Wordpress}

    \begin{itemize}
        \item Eshopy
        \item Aplikace se specifickym zamerenim
    \end{itemize}
\end{frame}


\begin{frame}
    {Instalace Wordpressu}

    % pass
\end{frame}


\begin{frame}
    {Prihlaseni}

    Pro pristup do administrace sveho blogu zadejte do prohlizece adresu \url{http://myblog.cz/wp-admin/}
\end{frame}

\begin{frame}
    {Dashboard}

    Tady se nachazeji vsechny nejcasteji pouzivane funkce.

    \begin{itemize}
        \item Informace o strankach a postech
        \item Vytvorit prispevek
        \item Posledni aktivita
    \end{itemize}
\end{frame}


\begin{frame}
    {Settings}

    % pass
\end{frame}


\begin{frame}
    {General Settings}

    Zakladni nastaveni blogu, nazev, url, formaty casu a data, ...
\end{frame}


\begin{frame}
    {Permalink Settings}

    Nastaveni jak maji vypadat url - (\url{http://myblog.cz/?p=123}, \url{http://myblog.cz/2015/02/18/sample-post/})
\end{frame}


\begin{frame}
    {Ostatni nastaveni}

    \begin{itemize}
        \item Writing Settings
        \item Reading Settings
        \item Discussion Settings
        \item Media Settings
    \end{itemize}
\end{frame}


\begin{frame}
    {Pages}

    \begin{itemize}
        \item Seznam vsech stranek
        \item Editace - vytvareni, mazani
        \item Rychla editace
    \end{itemize}
\end{frame}

\begin{frame}
    {Create/Edit Page}

    \begin{itemize}
        \item Title, Body
        \item Save draft, Preview, Publish/Update
        \item Permalink
    \end{itemize}
\end{frame}


\begin{frame}
    {Posts}

    \begin{itemize}
        \item Seznam postu
        \item Editace - vytvareni, mazani
        \item Rychla editace
        \item Kategorie
        \item Tagy
    \end{itemize}
\end{frame}


\begin{frame}
    {New/Edit Post}

    \begin{itemize}
        \item Title, Body
        \item Save draft, Preview, Publish/Update
        \item Format, Categories, Tags
    \end{itemize}
\end{frame}


\begin{frame}
    {Appereance}

    % pass
\end{frame}


\begin{frame}
    {Themes}

    \begin{itemize}
        \item Prepinani sablon
        \item Instalace sablony
    \end{itemize}
\end{frame}


\begin{frame}
    {Instalace}

    \begin{itemize}
        \item Vybranim ze seznamu - nutne ftp pristupove udaje
        \item Nahranim .zip souboru
    \end{itemize}
\end{frame}


\begin{frame}
    {Customize}

    Zde se da nastavit vetsina veci ohledne vzhledu blogu a konkretni sablony.

    \begin{itemize}
        \item Nazev blogu
        \item Obrazek hlavicky, pozadi (pokud to sablona umoznuje)
        \item Barevne varianty
        \item Menus
        \item Widgety
    \end{itemize}
\end{frame}


\begin{frame}
    {Menus}

    Tato sekce je specificka u kazde sablony ale zakladni funkce je stejna.

    \begin{itemize}
        \item Editace polozek v menu - odkazy, kategorie, stranky
        \item Nastaveni zobrazeni menu - pozice, typ, ...
    \end{itemize}
\end{frame}


\begin{frame}
    {Users}

    \begin{itemize}
        \item Seznam uzivatelu
        \item Pridani noveho uzivatele
    \end{itemize}
\end{frame}


\begin{frame}
    {Profile}

    Nastaveni daneho uzivatele: jmeno, heslo, vzhled admininstrace, ...
\end{frame}


\begin{frame}
    {Comments}

    \begin{itemize}
        \item Nastaveni komentaru - Settings -> Discussion
        \item Povolovani, Editace, Mazani
    \end{itemize}
\end{frame}


\begin{frame}
    {Plugins}

    \begin{itemize}
        \item Seznam instalovanych pluginu
        \item Aktivace / Deaktivace
        \item Instalace pluginu
    \end{itemize}
\end{frame}


\begin{frame}
    {Instalace}

    \begin{itemize}
        \item Vybranim ze seznamu - nutne ftp pristupove udaje
        \item Nahranim .zip souboru
    \end{itemize}
\end{frame}


\begin{frame}
    {Dekuji za pozornost}

    % pass
\end{frame}


\begin{frame}
    {Otazky}
\end{frame}

\end{document}

